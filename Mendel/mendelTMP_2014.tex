\documentclass[a4paper,10pt]{article}

%\usepackage{natbib}
\usepackage{amsthm}
\usepackage{amsfonts}
\usepackage{amssymb}
\usepackage{amsmath}
\usepackage{latexsym}
\usepackage{graphicx}

\usepackage{doc}

\newtheorem*{theorem}{Theorem}
\theoremstyle{definition}
\newtheorem*{definition}{Definition}

\hoffset -1in \topmargin 0mm \voffset 0mm \headheight 0mm
\headsep0mm
\oddsidemargin  20mm     %   Left margin on odd-numbered pages.
\evensidemargin 20mm     %   Left margin on even-numbered pages.
\textwidth   170mm       %   Width of text line.
\textheight  252mm

\makeatletter
\renewcommand\@openbib@code{%
     \advance\leftmargin  \z@ %\bibindent
      \itemindent \z@
     % Move bibitems close together
     \parsep -0.8ex
     }
\makeatother

\makeatletter
\renewcommand\section{\@startsection {section}{1}{\z@}%
                                   {-3.5ex \@plus -1ex \@minus -.2ex}%
                                   {1.5ex \@plus.2ex}%
                                   {\large\bfseries}}
\makeatother

\makeatletter
\renewcommand\subsection{\@startsection {subsection}{1}{\z@}%
                                   {-3.5ex \@plus -1ex \@minus -.2ex}%
                                   {1.5ex \@plus.2ex}%
                                   {\normalsize\bfseries}}
\makeatother

\makeatletter
	\setlength{\abovecaptionskip}{3pt}   % 0.25cm 
	\setlength{\belowcaptionskip}{3pt}   % 0.25cm 
\makeatother

\begin{document}
\pagestyle{empty}

\begin{center}
{\bf \Large PREPARATION OF THE PAPER FOR CONFERENCE\\
\medskip PROCEEDINGS (ALL CAPS, 14PT, BOLD, CENTERED)}
\end{center}

\smallskip
\begin{center}
{\large Authors' Names (Name Surname) (12pt Times New Roman, centered)}
\end{center}

\smallskip
\begin{center}
University (10pt Times New Roman, centered)\\
Department (10pt Times New Roman, centered)\\
Address (10pt Times New Roman, centered)\\
Country (10pt Times New Roman, centered)\\
youremailaccount@xxx.xxx.xx (10pt Times New Roman, centered)
\end{center}

\bigskip
\noindent Abstract: \textit{This is a sample of the format of your
full paper. Use Word for Windows (Microsoft) (or equivalent Word
Processor with exactly the same ``printing results'') or \LaTeX{} by
tuning -- A4 sheet, 20 mm from right, left and above, and 25 mm
below. Please do not number the pages! Use single space. Use 10pt, Times New
Roman for MS Word or Computer Modern font for \LaTeX{}. For text in abstract and keywords use Italics, 10pt. Leave one blank line after the Abstract.}

\vspace*{10pt} \noindent Keywords: \textit{Write your keywords (6--10 words).
Leave double blank line after your keywords.}

\bigskip
\section{Introduction}
\label{sec:1}
As you can see for the title of the paper you must use 14pt,
Capital, Centered, Bold. \textit{Please do not
number the pages!} Leave one blank line (14pt) and then type
Authors' Names etc., see above.

\textbf{Paper text should be typed in 10pt Times New Roman} for MS Word \textbf{or Computer Modern} for \LaTeX{}, and justify to
block. The heading of each section should be printed in small,
12pt, left justified, bold, serif. You must use the Arabic numbers
1, 2, 3,\,\dots for the sections numbering, not the Roman
numbers (I, II, III,\,\dots).

\section{Problem Formulation (Equations)}
\label{sec:2}
Please, leave two blank lines between successive sections as here (see Sect.~\ref{sec:1} to Sect.~\ref{sec:2}). Please note that the first line of text that follows a heading is not indented, whereas the first lines of all subsequent paragraphs are. Further on please use the
\LaTeX{} or MS Word (equivalent) automatism for all your cross-references and citations.

Mathematical equations must be centered and numbered as follows:~\eqref{eq:01},~\eqref{eq:02},$\ldots$, (99) and not (1.1), (1.2),$\ldots$, (2.1),
(2.2),$\ldots$ depending on your various Sections.
\begin{equation}z^{EO}=\min\limits_{e,\,g(\xi)}\mathbb{E}(F(\xi,e,g(\xi))),
\label{eq:01}
\end{equation}
\begin{equation} a_{min}\leq a\leq a_{max}.
\label{eq:02}
\end{equation}

\subsection{Subsection}
\label{subsec:1}
When including a subsection you must use, for its heading, small
letters, 10pt, left justified, bold as here.
Use the standard \verb|equation| environment to typeset your equations, however, for multiline equations we recommend to use the \verb|eqnarray| environment (\LaTeX{} users).

%\begin{eqnarray}
%a \times b = c \nonumber\\
%\mathbf{a} \cdot \mathbf{b}=\mathbf{c}
%\label{eq:01}
%\end{eqnarray}

\begin{definition}
Let $H$ be a subgroup of a group~$G$.  A \emph{left coset}
of $H$ in $G$ is a subset of $G$ that is of the form $xH$,
where $x \in G$ and $xH = \{ xh : h \in H \}$.
Similarly a \emph{right coset} of $H$ in $G$ is a subset
of $G$ that is of the form $Hx$, where
$Hx = \{ hx : h \in H \}$
\end{definition}

\begin{theorem}
This is a theorem content. Theorem text goes here. 
\end{theorem}

\begin{proof}
Let $z$ be some element of $xH \cap yH$.  Then $z = xa$
for some $a \in H$, and $z = yb$ for some $b \in H$.
If $h$ is any element of $H$ then $ah \in H$ and
$a^{-1}h \in H$, since $H$ is a subgroup of $G$.
But $zh = x(ah)$ and $xh = z(a^{-1}h)$ for all $h \in H$.
Therefore $zH \subset xH$ and $xH \subset zH$, and thus
$xH = zH$.  Similarly $yH = zH$, and thus $xH = yH$,
as required.
\end{proof}

\section{Problem Solution}
Figures\footnote{If you copy text passages, figures, or tables from other works, you must obtain \textit{permission} from the copyright holder (usually the original publisher or author). Please enclose the signed permission with the manuscript.} and Tables should be numbered as follows: Fig.~1,
Fig.~2,\,\dots{} etc. (see Fig.~\ref{fig:1}), Table 1, Table 2,\,\dots{} etc. (see Table~\ref{tab:1}). The figures are expected to be printed in colour (the text and tables strictly in black), but authors are strongly recommended to test the readability of the figures in gray shades to be on the safe side. Figure quality must be appropriate for the print and labels must be readable, our suggestion is resolution 300dpi and vector format is preferred. The screen capture bitmap in the case of graphs or diagrams is considered as highly inappropriate. Figure caption must be placed below the figure and table caption must be placed above the table.

If your paper deviates significantly from these specifications, our Mendel publishing house may not be able to include your paper in the Proceedings. When citing references in the text, type the corresponding number in square brackets as shown in the following two sentences.
This sentence refers to the article \cite{article-full}, the book \cite{book-full}, and to the article
in a proceedings \cite{inproceedings-full}, and to the online presentation \cite{online-misc}.
Combined citations like \cite{article-full,book-full} may have special appearance. \BibTeX{} users should use a Springer \BibTeX{} style \texttt{spmpsci.bst}.

%
% For figures use (also *.eps, *.tiff)
%
\begin{figure}[h]
\begin{center}
\includegraphics[scale=0.42]{fig.png}
\caption{Please write your figure caption here}
\label{fig:1}
\end{center}
\end{figure}


%
% For tables use
%
\begin{table}[h] 
\begin{center}
\caption{Please write your table caption here} 
\label{tab:1}
\begin{tabular}{lll}
\hline\noalign{\smallskip}
Parameter & Value& GATE implementation\\
\noalign{\smallskip}
\hline\noalign{\smallskip}
GA test suite & $F_{6}$, $5$ optimized
variables & funName: 'F6', nParam: 5,$\ldots$ \\
%GA selection & elite tournament selection,
%3& \textbf{select}(GA,'tournamentE',3)\\
%GA crossover& 80\% of individual in 3 cuts& \textbf{cross}(GA,'pcross',[0.8 3])\\
%GA mutation & mutation probability $0.02$& \textbf{mutation}(GA,'bitmut',0.02)\\
GAHC& $10$ HCA kernels of size $5$ bits& \textbf{mutationHC}(GA,'HC12',10,'rand',5)\\
\hline
\end{tabular}
\end{center}
\end{table}

\section{Conclusion}
Please, follow our instructions faithfully; otherwise you have to
resubmit your full paper. This will enable us to maintain
uniformity in the conference proceedings. The better you look, the
better we all look. We also encourage you to add the references to relevant articles from previous MENDEL conferences, it would be helpful for the conference as such and most importantly for the authors.
Thank you for your cooperation and
contribution. We are looking forward to seeing you at the Mendel
conference in Brno.

\vspace*{10pt} \noindent {\bf Acknowledgement:} On this place you
can return thanks for the support. Use 10pt Times New Roman.

% References
%
\begingroup
\makeatletter
\renewcommand\section{\@startsection {section}{1}{\z@}%
                                   {-3.5ex \@plus -1ex \@minus -.2ex}%
                                   {4.5ex \@plus.2ex}%
                                   {\large\bfseries}}
\makeatother

\begin{thebibliography}{99}
\providecommand{\url}[1]{{#1}}
\providecommand{\urlprefix}{URL }
\expandafter\ifx\csname urlstyle\endcsname\relax
  \providecommand{\doi}[1]{DOI~\discretionary{}{}{}#1}\else
  \providecommand{\doi}{DOI~\discretionary{}{}{}\begingroup
  \urlstyle{rm}\Url}\fi

\bibitem{book-full}
Knuth, D.E.: Seminumerical Algorithms, \emph{The Art of Computer Programming},
  vol.~2, second edn.
\newblock Addison-Wesley, Reading, Massachusetts (1981)

\bibitem{article-full}
Matousek, R., Zampachova, E.: Promising {GAHC} and {HC12} algorithms in global
  optimization tasks.
\newblock Optimization Methods and Software \textbf{26}(3), 405--419 (2011).
\newblock \doi{10.1080/10556788.2011.556826}

\bibitem{inproceedings-full}
Klapka, J., Matousek, R., Sevcik, V.: Improvement of time-periodical production
  schedule of the group of products in the group of workplaces through the lot
  sizes alteration.
\newblock In: R.~Matousek (ed.) Proceedings of 17th International Conference on
  Soft Computing -- MENDEL 2011, no.~17 in MENDEL, pp. 334--340. Brno
  University of Technology, VUT Press, Brno (2011)

\bibitem{online-misc}
Osmera, P., Werner, P.: My Rings.
\newblock \url{http://youtu.be/jVfKTU168QA} (2013).
\newblock [Online; accessed 16-May-2013] 

\end{thebibliography}

%\bibliography{xampl}{}
%\bibliographystyle{spmpsci}
\endgroup

\end{document}
