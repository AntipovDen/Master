\documentclass[a4paper,10pt]{article}
\usepackage[utf8]{inputenc}

\usepackage[russian]{babel}
\usepackage{amsmath}


\begin{document}

В алгоритме генерации тестов для алгоритма Дейкстры (в случае с дискретными длинами ребер) рассмотрим последнюю фазу алгоритма на графе с двумя вершинами, когда он пытается релаксировать самое последнее ребро. Рассмотрим наихудший случай, когда $M =E$, где $M$ -- максимальная длина ребра, $E$ -- число ребер графа.

Алгоритм может находиться в одном из следующих состояний:
\begin{itemize}
 \item Unstable state (US) -- когда релаксированное ребро имеет такую же длину, как и предыдущее (заметим, что это состояние невозможно, если нерелаксированным является самое первое ребро). Особенность этого состояния в том, что позиция нерелаксированного ребра в списке ребер может уменьшиться.
 \item Potentially unstable state (PUS) -- когда релаксированное ребро имеет длину на 1 больше, чем предыдущее, при этом незанятая дистанция находится до нерелаксированного ребра. В данном состоянии алгоритм почти всегда ведет себя также, как и в Stable state, кроме особого случая, когда позиция нерелаксированного ребра ровно на один больше, чем позиция свободной длины.
 \item Stable state (SS) -- все остальные случаи.
\end{itemize}

Далее $i$ обозначает позицию нерелаксированного ребра, а $j$ -- позицию свободной длины. Для избежания недопонимания, уточним, как считаются эти позиции: если вернуться к нашей старой терминологии $d_k$ и $\delta_k$, то оба эти массива имеют на жанном этапе алгоритма $E - 1$ элемент, причем только одна $d_k = 1$ и только одна $\delta_k = 2$. Остальные $d_k = 0$, а остальные $\delta_k = 1$. В данной терминологии $i$ такое, что $d_i = 1$, а $j$ такое, что $\delta_j = 2$.

Тут, наверное, стоит описать все сложное поведение вероятностей переходов между состояниями и сдвигов $i$ и $j$. Рассматриваются отдельно следующие случаи (думаю, я смогу нарисовать наглядные иллюстрации для каждого, но потом):
\begin{itemize}
 \item $|i - j| > 1$
 \begin{itemize}
  \item $i > j$
  \begin{itemize}
   \item SS
   \item US
   \item PUS
  \end{itemize}
  \item $i < j$
  \begin{itemize}
   \item SS
   \item US
  \end{itemize}
 \end{itemize}
 \item $ i = j + 1$
  \begin{itemize}
   \item SS
   \item US
   \item PUS
  \end{itemize}
 \item $i = j$
  \begin{itemize}
   \item SS
   \item US
  \end{itemize}
 \item $i = j - 1$
  \begin{itemize}
   \item SS
   \item US
  \end{itemize}
\end{itemize}

Теперь введем следующие обозначения:
\begin{itemize}
 \item $T(i,j)$ -- матожидание числа итераций алгоритма до его завершения в SS и позициями нерелаксированного ребра и свободной дистаниции $i$ и $j$ соответственно.
 \item $T^{us}(i, j)$ -- то же самое для US. При этом заметим, что для $i = 0$ этот значение неопределено.
 \item $T^{pus}(i, j)$  -- для PUS. Не существует для $j \ge i$ (в том числе и для $i = 0$
\end{itemize}

Таким образом, у нас есть $E^2 + E(E - 1) + \frac{E(E - 1)}{2} = \frac{E(5E - 3)}{2}$ неизвестных матожиданий. Однако мы можем составить систему линейных уравнений для них:

 \begin{align*}
  T(i,j) =& (j \ne 0) \frac{1}{2E^2} (T(i, j - 1) + 1) + (j \ne E - 1) \frac{1}{2E^2} (T(i, j + 1) + 1) + \\
         & (i \ne 0) \frac{1}{2E^2} (T^{us}(i, j) + 1) + (i > j) \frac{1}{2E^2} (T^{pus} (i, j) + 1) + \\
         & (i \ne E - 1) \frac{1}{2E^2} (T^{us} (i + 1, j) + 1) + (i = j)\frac{1}{2E^2} \cdot 1 +  \\
         & (i = j - 1) \frac{1}{2E^2} (T^{pus}(i + 1, j - 1) + 1) + p_{left} (T(i, j) + 1) \\
 \end{align*}
 
 \begin{align*}
  T^{us}(i,j) =& (j \ne 0 \& i \ne j) \frac{1}{2E^2} (T^{us}(i, j - 1) + 1) + (i = j)\frac{1}{2E^2}(T^{pus}(i, j - 1) + 1) +\\
               & (j \ne E - 1 \& i \ne j + 1) \frac{1}{2E^2} (T^{us}(i, j + 1) + 1) + (j \ne E - 1 \& i = j + 1) \frac{1}{2E^2} \cdot 1 \\
               & (i \ne E - 1) \frac{1}{2E^2} (T^{us} (i + 1, j) + 1) + \frac{E + i - 1}{2E^2}(T(i, j) + 1) +  \\
               & (i > 1) \frac{1}{2E^2} (T^{us}(i - 1, j) + 1) + (i > j) \frac{1}{2E^2} (T^{pus} (i, j) + 1) + \\
               & (i > 1) \frac{E + i - 2}{2E^2}(T(i - 1, j) + 1) + (i = 1)\frac{E + i - 1}{2E^2}(T(i - 1, j) + 1) \\
               & (i > j + 1) \frac{1}{2E^2} (T^{pus}(i - 1, j) + 1) + (i = j) \frac{1}{2E^2} \cdot 1 \\
               & (i = j - 1) \frac{1}{2E^2} (T^{pus}(i + 1, j - 1) + 1) + p_{left} (T^{us}(i, j) + 1) \\
 \end{align*}
 
 \begin{align*}
  T^{pus}(i,j) =& (j \ne 0) \frac{1}{2E^2} (T^{pus}(i, j - 1) + 1) + (j \ne i - 1) \frac{1}{2E^2} (T^{pus}(i, j + 1) + 1) + \\ 
               & (j = i - 1) \frac{1}{2E^2} (T^{us}(i, j + 1) + 1) + (i \ne E - 1) \frac{1}{2E^2} (T^{us} (i + 1, j) + 1) + \\ 
               & \frac{1}{2E^2} (T^{us} (i, j) + 1) + \frac{E + i - 1}{2E^2}(T(i, j) + 1) +  \\
               & (i = j + 1 \& i \ne 1) \frac{1}{2E^2} (T^{us}(i - 1, j + 1) + 1) + (i = j + 1 \& i \ne 1) \frac{E + i - 2}{2E^2}(T(i - 1, j + 1) + 1) + \\ 
               & (i = 1 \& j = 0)\frac{E + i - 1}{2E^2}(T(i - 1, j + 1) + 1) p_{left} (T^{pus}(i, j) + 1)\\
 \end{align*}


\end{document}
