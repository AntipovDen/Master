\documentclass{spisok-article}

\title{Теоретический анализ времени работы эволюционных алгоритмов при генерации тестов
}

\author{
  Антипов Д. С.,
  программист кафедры ТП университета ИТМО,
  antipovden@yandex.ru
  Буздалов М. В.,
  доцент кафедры КТ университета ИТМО,
  mbuzdalov@gmail.com
}

\begin{document}

\maketitle

\begin{abstract}
Эволюционные алгоритмы успешно использовались для генерации тестов олимпиадных задач~\cite{max}.
Однако теоретическая оценка времени их работы на данный момент не была выполнена.
В данной статье приводятся результаты анализа ожидаемого времени работы эволюцинного алгоритма,
генерирующего тест для алгоритма Дейкстры, на котором тестируемая реализация будет релаксировать все ребра.
\end{abstract}

\section{Введение}



\section{Формат конференции (заголовок I уровня)}

Формат конференции подразумевает выступление с интересным и
содержательным докладом, по итогам доклада рекомендуются к публикации
в сборнике конференции тезисы, в отношении которых справедливо:

\begin{itemize}
\item
  текст содержит:
  \begin{enumerate}
  \item
    аннотацию;
  \item
    введение;
  \item
    один или несколько разделов;
  \item
    заключение;
  \item
    список литературы;
  \end{enumerate}
\item
  объём текста (всего кроме списка литературы) от 650 до 1250 слов;
  если это требование нарушено, то решение о включении тезисов в
  сборник принимает программный комитет, опираясь, в основном, на
  мнение руководителя секции, на которой прозвучал доклад;
\item
  основной язык конференции русский, однако члены программного
  комитета могут (и будут стараться) приглашать иностранных
  докладчиков, тезисы докладов которых могут публиковаться
  по-английски; публикации на прочих языках отдельно согласуются с
  программным комитетом.
\end{itemize}

\section{Форматирование тезисов (заголовок I уровня)}

\subsection{Основной текст (заголовок II уровня)}

Основной текст тезисов отформатирован следующим образом:

\begin{enumerate}
\item
  шрифт Times New Roman\footnote{Для Xe\LaTeX ~действительно
  джентльменский набор из Times, \textsf{Arial} и \texttt{Courier},
  а у PDF\LaTeX ~с кириллицей в Times объективные трудности, поэтому
  PDF\LaTeX ~использует шрифты CM.  В итоге, чистовая вёрстка
  выполняется при помощи Xe\LaTeX, будьте к этому готовы!}, кегль 10
\item
  первая страница: Все поля по 17 мм
\item
  последующие страницы:

  \begin{enumerate}
  \item
    все поля, кроме верхнего, по 17 мм;
  \item
    верхнее поле 23;
  \end{enumerate}
\item
  аннотация имеет дополнительные отступы по 10 мм слева и справа.
\end{enumerate}

\subsection{Цитаты, врезки изображений (заголовок II уровня)}

Ниже процитирован отрывок из Метафизики Аристотеля. Отметим, что
данная цитата несёт некоторую смысловую нагрузку и в контексте данного
документа, показывая, что цитаты следует выделять курсивом.

\emph{\ldots{} В самом деле, определенное умение читать и писать
  принадлежит к тому, что находится в подлежащем, но ни о каком
  подлежащем не говорится как об определенном умении читать и
  писать)\ldots}

\begin{figure}[h]
\begin{center}
\includegraphics[width=0.3\textwidth]{Aristotle.jpg}
\end{center}
\caption{Аристотель глазами составителей Нюрнбергской хроники,
  1493}\label{fig:aristotle}
\end{figure}

Добавим лишь, что Аристотель в Нюрнбергской хронике
(см. Рис.~\ref{fig:aristotle}) был изображён в цвете, но в XXI веке
твёрдые копии сборников большинства конференций этим похвастаться не
могут. Поэтому в отношении всех цветных иллюстраций очень желательно
удостовериться в том, что и в чёрно-белом виде они не потеряют смысла.

\subsection{Прочие врезки и ссылки (заголовок II уровня)}

При врезке графиков и диаграмм следует придерживаться тех же правил,
что и при врезке изображений. Отдельно настоятельно рекомендуется
графики и диаграммы врезать, используя векторные форматы изображений,
так как это, опять же, более предсказуемо выглядит в твёрдой копии.

При наборе формул просьба по возможности использовать встроенные
средства офисного пакета.

Фрагменты текстов программ следует вставлять при помощи пакета
\texttt{listing}:

\begin{lstlisting}[language=C]
int main()
{
    return 0;
}
\end{lstlisting}

Библиографическию следует оформлять стандартными средствами
\texttt{\textbackslash{}cite} и
\texttt{\textbackslash{}bibitem}~\cite{medvedev2011}. BibTeX, BibLaTeX
и подобные средства хорошо работают в собственных руках на собственных
текстах, но, попав в чужие, делают сюрпризы.  Поэтому просьба либо их
не использовать, либо использовать так, чтобы организаторы конференции
об этом не знали.

\section{Заключение}

В документе были представлены основные стили текста и макросы, которые
могут быть использованы при форматировании тезисов конференции СПИСОК.
Собственные тезисы рекомендуется набирать в этом документе, заменяя
текст и заголовки на свои.

\renewcommand\refname{Литература}
\begin{thebibliography}{8}

\bibitem{medvedev2011} Медведев О. Use case: отладка реализации RISC
  процессора для FPGA // % Материалы 2-й межвузовской научной
  конференции по проблемам информатики<<СПИСОК-2011>>. --- % 2011. ---
  С. 7--12.
  \href{http://spisok.math.spbu.ru/txt/SPISOK-2011.pdf}{http://spisok.math.spbu.ru/txt/SPISOK-2011.pdf}

\end{thebibliography}

\end{document}
