\documentclass{spisok-article}

\title{Теоретический анализ времени работы эволюционных алгоритмов при генерации тестов
}

\author{
  Антипов Д. С.,
  программист кафедры ТП университета ИТМО,
  antipovden@yandex.ru
  Буздалов М. В.,
  доцент кафедры КТ университета ИТМО,
  mbuzdalov@gmail.com
}

\begin{document}

\maketitle

\begin{abstract}
Эволюционные алгоритмы успешно использовались для генерации тестов олимпиадных задач~\cite{max}.
Однако теоретическая оценка времени их работы на данный момент не была выполнена.
В данной статье приводятся результаты анализа ожидаемого времени работы эволюцинного алгоритма,
генерирующего тест для алгоритма Дейкстры, на котором тестируемая реализация будет релаксировать все ребра.
\end{abstract}

\section{Введение}
Эволюционные алгоритмы -- это класс алгоритмов оптимизации. Они используют идеи, взятые из биологической эволюции. В общем виде их можно описать следующим образом:
\begin{itemize}
 \item Берется начальное множество (поколение) возможных решений.
 \item На каждой итерации алгоритма происходит генерация нового покления с помощью скрещивания и мутации особей предыдущего поколения.
 \item Алгоритм продолжает работу пока не будет найден оптимум или достаточно хорошее решение.
\end{itemize}
Эволюционные алгоритмы не всегда способны найти наилучшее решение, однако они способны находить достаточно хорошие решения за гораздо меньшее время, чем детерменированные алгоритмы оптимизации. 
Именно поэтому были разработаны эволюционные алгоритмы для генерации тестов для олимпиадных задач~\cite{max}. Однако не было проведено никаких теоретических исследований времени их работы. Заметим, что обычно под временем работы эволюционного алгоритма подразумевают количество запросов этого алгоритма к оптимизируемой функции или число его итераций.
В качестве первого представителя данного класса эволюционных алгоритмов для теоретической оценки ожидаемого времени работы был выбран эволюционный алгоритм, генерирующий тест для алгоритма Дейкстры, на котором данная реализация тестируемого алгоритма будет релаксировать все ребра графа. Ниже представлен псевдокод этого алгоритма. 
В качестве оператора мутации используется изменение случайно выбранного ребра случайным образом.
Алгоритм Дейкстры в данном случае возвращает не кратчайшие расстояния до всех вершин, а только число релаксированных ребер.

\begin{algorithm}[H]
  \KwData{$V$, $E$}
  \KwResult{$graph$ with all the edges relaxed}
  
  $graph \gets init()$
  
  $f \gets dijkstra(graph)$
  
  \While{$f \ne E$}{
    
    $graph' \gets mutate(graph)$
    
    $f' \gets dijkstra(graph')$
    
    \If{$f' \ge f$}{
      $graph, f \gets graph', f'$
    }
  }
 % \Return{$graph$}
  \end{algorithm}


\section{Теоретический анализ времени работы}

\subsection{Разреженный граф}
В случае, когда число вершин $V$ много больше числа ребер $E$, можно сказать, что исследуемый алгоритм работает не медленнее, чем его модификация, которая не принимает новый граф в случае, когда нарушается инвариант: все установленные ребра представляют собой дерево с корнем в начальной вершине. Время работы $T$ такой модификации оценить гораздо проще:
\begin{itemize}
    \item Если $V > E + 1$:
    $$T \le \frac{EV(2V - E - 1)}{(E + 1)(V - E - 1}(\gamma + \ln E) - \frac{EV}{V - E - 1} \ln\frac{V - 1}{V - E}$$
    где $\gamma$ -- константа Эйлера
    \item Если $V = E + 1$:
    $$T \le \frac{\pi^2 E^2}{6} + o\left( \frac{\pi^2 E^2}{6} \right)$$
\end{itemize}
Но эти оценки работают только для $V > E$, иначе время работы становится бесконечно большим.

Также заметим, что если $V = aE$, где $a$ это некоторая константа больше единицы, тогда время работы асимптотически при $E \to +\infty$ можно ограничить следующим образом: $T \le V \ln{E} \left(2 + 1/(a - 1)\right)$.

\subsection{Плотный граф}

\section{Заключение}


\renewcommand\refname{Литература}
\begin{thebibliography}{8}

\bibitem{max} Диссертация на тему «Генерация тестов для определения неэффективных решений олимпиадных задач по программированию с использованием эволюционных алгоритмов»

\end{thebibliography}

\end{document}
