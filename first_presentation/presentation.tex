\documentclass{beamer}
\usetheme{CambridgeUS}

%\documentclass[handout]{beamer}
%\usetheme{Pittsburgh}
%\beamertemplatesolidbackgroundcolor{black!2}
%\setbeamertemplate{footline}[frame number]
%\usepackage{pgfpages}
%%\pgfpagesuselayout{4 on 1}[a4paper,border shrink=5mm,landscape]
%\pgfpagesuselayout{8 on 1}[a4paper,border shrink=5mm]

%%% PACKAGES
\usepackage[russian]{babel}
\usepackage[utf8]{inputenc}
\usepackage{amsmath}
\usepackage{amssymb}
\usepackage{tikz}
\usepackage{graphics}
\usepackage{color}

%%% BEAMER SETTINGS
\setbeamertemplate{navigation symbols}{}
\setbeamertemplate{headline}{}

%%% TIKZ SETTINGS
\usetikzlibrary{fit}

%%% NEW COMMANDS
\def\pitem{\pause \item}

\DeclareGraphicsRule{.1}{mps}{*}{}
\DeclareGraphicsRule{.2}{mps}{*}{}
\DeclareGraphicsRule{.3}{mps}{*}{}
\DeclareGraphicsRule{.4}{mps}{*}{}
\DeclareGraphicsRule{.5}{mps}{*}{}
\DeclareGraphicsRule{.6}{mps}{*}{}

%\includeonlyframes{current} % leaves only the given frames

\title[Генерация тестов]{Теоретический анализ времени работы эволюционных алгоритмов при генерации тестов}
%\transduration{20}
\author[Денис Антипов]{Денис Антипов}
\institute[Университет ИТМО]{Национальный исследовательский университет информационных технологий, механики и оптики}
\date{}

\begin{document}

\begin{frame}
%\transduration{20}
\begin{center}
{\scriptsize Санкт-Петербургский национальный исследовательский университет \\ информационных технологий, механики и оптики}

\vspace{1cm}

{\scriptsize Факультет информационных технологий и программирования

Кафедра компьютерных технологий}

\vspace{1cm}

\vbox{\large\bfseries
Теоретический анализ времени работы эволюционных алгоритмов при генерации тестов
}

\vspace{1cm}

{\large Антипов Денис Сергеевич \\}
{\large Группа 5539}


\vspace{1cm}

{\large Научный руководитель: к.т.н., ассистент кафедры ТП М.~B.~Буздалов}


\end{center}
\end{frame}

%%%%%%%%%%%%%%%%%%%%%%%%%%%%%%%%%%%%%%%%%%%%%%%%%%%%%%%%%%%%%%%%%%%%%%%%%%%%%%%%%%%%%%%%%%%%%%%%%%%%%%%%%%%%%%%%%%%%%%%%
%\frame[label=title]{\titlepage}

\begin{frame}{Решаемая проблема}
%\transduration{20}
\begin{block}{Алгоритм Дейкстры}
\begin{itemize}
\item Алгоритм на графах, находящий кратчайшие расстояния от одной вершины графа до всех остальных.
\item Работает только на графах без ребер отрицательного веса.
\end{itemize}
\end{block}
\end{frame}

\begin{frame}{Решаемая проблема}
%\transduration{20}
\begin{block}{Генерация тестов}
\begin{itemize}
\item Для генерации графа используется эволюционный алгоритм.
\item Максимизируется число релаксируемых ребер.
\item Начальный граф: все ребра из одной вершины в началаьную.
\item На каждой итерации случайным образом переставляем случайное ребро.
\item Берется новый граф, если число релаксируемых ребер в нем не меньше. Иначе остается предыдущий.
\end{itemize}
\end{block}
\end{frame}

\begin{frame}{Текущие результаты}
%\transduration{20}
\begin{block}{Среднее число итераций}
При фиксированном числе вершин: \textcolor{blue}{5}, \textcolor{red}{10} и \textcolor{green}{15} (200 запусков алгоритма):
\begin{center}
\includegraphics*[height=5cm]{pic/const_vertexes_20.png}
\end{center}
\end{block}
\end{frame}

\begin{frame}{Текущие результаты}
%\transduration{20}
\begin{block}{Среднее число итераций}
При фиксированном числе ребер: \textcolor{blue}{10}, \textcolor{red}{20} и \textcolor{green}{30} (200 запусков алгоритма):
\begin{center}
\includegraphics*[height=5cm]{pic/const_edges.png}
\end{center}
\end{block}
\end{frame}

\begin{frame}{Текущие результаты}
%\transduration{20}
\begin{block}{Сбор статистики}
Были найдены средние величины, зависящие от номера итерации:
\begin{itemize}
 \item Число вершин в КС.
 \item Число листьев.
 \item Максимальная глубина (в ребрах и в весах).
 \item Число недостижимых ребер.
\end{itemize}
\end{block}
\end{frame}

\begin{frame}{Текущие результаты}
%\transduration{20}
\begin{block}{Исследование}
Существовала идея разбить время работы на 2 части:
\begin{itemize}
 \item Разрастание графа.
 \item Расстановка оставшихся нерелаксированных ребер.
\end{itemize}
\end{block}
\end{frame}


\begin{frame}{Текуще результаты}
%\transduration{20}
\begin{block}{Разрастание графа}
\begin{itemize}
 \item Компонента связности может уменьшаться с вероятностью, пропорциональной числу листев в графе.
 \item Ребра могут образовывать некоторые отдельные компоненты связности.
\end{itemize}
Без этих фактов получено выражение матожидания времени разрастания:
$M = \frac{VE}{(E + 1)(E + 1 - V)} \left( (2E + 2 - v) \ln (V - 1) - V \ln \frac{E}{E + 1 - V}\right)$
\end{block}
\end{frame}

\begin{frame}{Текуще результаты}
%\transduration{20}
\begin{block}{Разрастание графа}
$M = \frac{VE}{(E + 1)(E + 1 - V)} \left( (2E + 2 - v) \ln (V - 1) - V \ln \frac{E}{E + 1 - V}\right)$
\begin{center}
\includegraphics*[height=5cm]{pic/Formula.png}
\end{center}
\end{block}
\end{frame}

\begin{frame}{Текуще результаты}
%\transduration{20}
\begin{block}{Число вершин в КС}
\begin{center}
\includegraphics*[height=4cm]{pic/reachable_vertexes_v15e15.png}
\includegraphics*[height=4cm]{pic/reachable_vertexes_v15e30.png}
\end{center}
\end{block}
\end{frame}

\begin{frame}{Текуще результаты}
%\transduration{20}
\begin{block}{Матожидание числа листьев на $i$-й итерации}
\begin{itemize}
 \item Было придумано выражение для вероятности того, что вершина - лист на $i$-й итерации, являющееся суммой достаточно сложных 
выражений.
 \item Первое слагаемое - $\frac{r}{V^2}\left(\frac{V - 1}{V} \right)^{2(i-1)}\left( 1 - \left(\frac{E - 1}{E} \right)^{i - 1} \right)$
 \item $r$  - число вершин в компоненте связности на $i$-й итерации, берется из результатов эксперимента.
\end{itemize}
\end{block}
\end{frame}

\begin{frame}{Текуще результаты}
%\transduration{20}
\begin{block}{Сравнение выражения с реальным результатом}
\begin{center}
\includegraphics*[height=4cm]{pic/firstPart.png}
\includegraphics*[height=4cm]{pic/leaves_v15e15.png}
\end{center}
\end{block}
\end{frame}

\begin{frame}{Дальнейшие действия}
%\transduration{20}
\begin{block}{}
\begin{itemize}
\item Добить матожидание числа лисьев.
\item Оценить вероятность того, что ребро попало в релаксируемую позицию.
\end{itemize}
\end{block}
\end{frame}

\begin{frame}{}
\begin{center}
Спасибо за внимание!
\end{center}
\end{frame}



\end{document}
