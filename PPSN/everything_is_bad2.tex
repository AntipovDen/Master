\documentclass[a4paper]{article}


\usepackage[russian]{babel}
\usepackage[utf8]{inputenc}
\usepackage{graphicx}
\usepackage{amsmath}

\begin{document}
 Наш потенциал:
 $$\Phi = \sum_{i = 0}^f d_i \delta_i, \Phi_0 = E$$
 Потенциал для инварианта:
 $$\Psi = \sum_{i = 0}^f (d_i \delta_i - d_i(d_i + 1)2\delta) = \Phi - \sum_{i = 0}^f d_i(d_i + 1)2\delta$$
 $$\Psi_0 = E(1 - 2\delta - 2\delta E)$$
 Заметим, что $\Psi_0$ может быть подогнан к $\Phi_0$ сколь угодно близко с помощью $\delta$.
 Если рассмотреть матожидание изменения $\Psi$, то получим, что оно строго меньше матожидания изменения $\Phi$, но больше, чем $\frac{2}{3}$ от него.
 
 И дальше я не понимаю, что делать: вроде, получается, что $\Psi$ не занулится раньше $\Phi$, но только ожидаемо. К тому же положительность $\Psi$, как я понимаю, не означает сохранение инварианта $\delta_i \ge (d_i + 1)2\delta$, а только то, что вероятность сохранения инварианта не станет нулем. Хотя с другой стороны, вероятность сохранения инварианта есть отношние $\Psi$ к $\Phi$. Но следует ли из того, что они уменьшаются примерно одинаково из примерно одинаковых начальных значений, то, что сохраняется и их отношение? И как получить из этого вероятность того, что минимальный $\Phi$ будет не меньше $\delta$?
 
 У меня уже совсем каша в голове, я запутался и не понимаю(
 
 Также перепроверил хотя бы сходимость ряда матожиданий: чтобы он сходился, надо чтобы начиная с некоторого $k$ вероятность $p(\Phi_{min} < \frac{1}{E^{k - 2}})$ была не больше, чем $\frac{1}{k^\alpha E^k}$, где $\alpha > 2$ (строго). В этом случае ряд сойдется и время работы будет -- сумма первых его членов, для которых вероятность больше, а также $8E^2\ln E + 4\ln E 2^{2 - \alpha} \frac{1}{2 - \alpha}$, что уже больше похоже на правду.
 
\end{document}
