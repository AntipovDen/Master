\documentclass[a4paper]{article}


\usepackage[russian]{babel}
\usepackage[utf8]{inputenc}
\usepackage{graphicx}
\usepackage{amsmath}

\begin{document}
 Наш потенциал:
 $$\Phi = \sum_{i = 0}^f d_i \delta_i$$
 
 Рассмотрим аккуратнее матожидание его изменения:
 
 $$E(\Phi_t - \Phi_{t + 1}) = \sum_{i = 0}^f \sum_{j = 1}^{d_j} \frac{1}{E} \cdot \frac{1}{4} \cdot \delta_i \cdot \frac{1}{\delta_i} \int_0^{\delta_i} \left[ d_i\delta_i-(j - 1)(\delta_i - l) - (d_i - j - 1)l\right]dl$$
 
 Пояснение: берем сумму по всем интервалам $d_i$, потом по всем нерелаксированным ребрам внутри интервала. Слагаемые: вероятность выбрать данное ребро, умноженная на вероятность правильно его ориентировать, на вероятность попасть в нужный интервал, а потом на интеграл. Интеграл, по сути, является матожиданием изменения потенциала при выборе и правильной ориентации данного ребра.
 Проинтегрировав, получим, что слагаемые не зависят от $j$, а значит, сумма по $j$ сворачивается в $d_i$. Получаем:
 $$E(\Phi_t - \Phi_{t + 1}) = \sum_{i = 0}^f \frac{d_i(d_i + 2)\delta_i^2}{8E}$$
 Единственное критическое изменение -- $\delta_i$, на которую я забыл домножить раньше (вероятность попадания в допустимый интервал), но от этого теперь ограничения ломаются ($\delta$ -- нижняя граница $\Phi_{min}$):
 $$E(\Phi_t - \Phi_{t + 1}) \ge \frac{\delta}{4E} \Phi_t$$ 
 $$ T \le \frac{4E}{\delta} \left(1 + \ln\frac{E}{\delta}\right)$$
 Но потом я понял еще одну вещь (полезно пытаться рассказать все кому-нибудь, кто не в курсе дела): если мы вдруг выбрали ребро, которое раньше релаксировалось и поставили его в допуcтимый интервал, то потенциал также меняется. Причем если ребро стояло хорошо (с запасом удовлетворяя нашему инварианту), то оно с большой вероятностью встанет в худшую позицию и наоборот: из плохой позиции в хорошую. Таким образом, у разницы потенциалов появляется еще одно слагаемое, с которым все становится совсем плохо:
 $$E(\Phi_t - \Phi_{t + 1}) = \sum_{i = 0}^f \frac{d_i(d_i + 2)\delta_i^2}{8E} + \sum_{i = 1}^f \frac{1}{4E} \int_{-\delta_i}^{\delta_{i + 1}} (d_{i - 1} \delta_{i - 1} + d_i \delta_i - d_{i-1}(\delta_{i - 1} - l) - d_i(l + \delta_i)) dl = $$
 $$ = \sum_{i = 0}^f \frac{d_i(d_i + 2)\delta_i^2}{8E} + \sum_{i = 1}^f \frac{d_{i - 1} - d_i}{8E}(\delta_{i - 1}^2 - \delta_i^2)$$
 
На первый взгляд казалось, что у вторая сумма как-то телескопически сжимается, но это не так, по крайней мере на второй взгляд. С ней я совсем не понимаю что делать.
 
 
\end{document}
