\documentclass{beamer}
\usetheme{CambridgeUS}


%%% PACKAGES
\usepackage[russian]{babel}
\usepackage[utf8]{inputenc}
\usepackage{amsmath}
\usepackage{amssymb}
\usepackage{tikz}
\usepackage{graphics}
\usepackage{color}

%%% BEAMER SETTINGS
\setbeamertemplate{navigation symbols}{}
\setbeamertemplate{headline}{}

%%% TIKZ SETTINGS
\usetikzlibrary{fit}

%%% NEW COMMANDS
\def\pitem{\pause \item}


%\includeonlyframes{current} % leaves only the given frames

\begin{document}
\title[Автоматическая генерация тестов]{Теоретический анализ времени работы эволюционных алгоритмов при генерации тестов}
%\transduration{20}
\author[Денис Антипов]{Денис Антипов}
\institute[Университет ИТМО]{Национальный исследовательский университет информационных технологий, механики и оптики}
\date{13.04.2016}

 
\begin{frame}
 \maketitle
\end{frame}
 
 \begin{frame}{План презентации}
  \tableofcontents
 \end{frame}

 \section{Введение}
 \subsection{Эволюционные алгоритмы}
 \begin{frame}{Эволюционные алгоритмы}
  \begin{itemize}
   \item Алгоритмы оптимизации, исполбзующие идеи эволюции.
   \item При инициализации алгоритма создается первое поколение решений.
   \item На каждой итерации алгоритм генерирует новое поколение на основе предыдущего.
   \item Алгоритм заканчивает работу, когда находит достаточно хорошее решение.
  \end{itemize}
 \end{frame}
 
 \subsection{Генерация тестов}
 \begin{frame}{Генерация тестов}
  
 \end{frame}



\end{document}
