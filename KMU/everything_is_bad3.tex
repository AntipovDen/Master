\documentclass[a4paper]{article}


\usepackage[russian]{babel}
\usepackage[utf8]{inputenc}
\usepackage{graphicx}
\usepackage{amsmath}

\begin{document}
 
  Опять бага в разности потенциалов (на этот раз она совсем небольшая и ни на что фактически не влияет):
  $$E(\Phi_t - \Phi_{t + 1}) = \frac{1}{E} \sum_{i = 0}^f \frac{d_i(d_i+1)}{2} \delta_i^2$$
  Если что, единственное отличие -- то, что $d_i + 1$ было $d_i + 2$.
  Если вдруг предположить довольно странный инвариант $\delta_i (d_i + 1) \ge 2\delta$, то можно сказать, что 
  $$E(\Phi_t - \Phi_{t + 1}) \ge \frac{\delta \Phi_t}{4E}$$
  Странный он потому, что мы требуем, по сути, чтобы у маленькой $\delta_i$ была большая $d_i$, что, казалось бы плохо. Но если подумать, то чем этот инвариант ближе к пределу своего выполнения, тем больше вероятность его поломки.
  
  Я попытался оценить вероятность нарушения потенциала при условии его выполнения при релаксации, но из этого ничего особо не вышло:
  $$p(\exists \delta_i: \delta_i(d_i + 1) < 2\delta | \text{relax}, \forall \delta_i: \delta_i(d_i + 1) > 2\delta) = \frac{\frac{1}{E - f}\sum_{i = 0}^{f}\sum_{j = 1}^{d_i} 2\delta\left(\frac{1}{d - j + 1)} + \frac{1}{j} \right)}{p_{\text{relax}}} = $$
  $$ = \frac{16E\delta}{(E - f)\Phi}\sum_{i = 0}^f \sum_{j = 1}^{d_i} \frac{1}{j}$$
  На этой конструкции я уже застрял.
  
  Пытался еще просто оценить вероятность того, что $\delta_{min}$ станет меньше и матожидание ее изменение при релаксации, но там совсем дебри.
 
\end{document}